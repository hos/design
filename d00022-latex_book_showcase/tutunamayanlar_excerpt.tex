\documentclass[a5paper,13pt]{scrbook}

\usepackage[T1]{fontenc}
\usepackage[turkish]{babel}
\usepackage[utf8]{inputenc}

\addtokomafont{disposition}{\rmfamily}

\usepackage{microtype}

% \newcommand{\sectionbreak}{\clearpage}
% \addtokomafont{section}{\clearpage\centering}
% \addtokomafont{chapter}{\clearpage\centering}

% \RedeclareSectionCommand[
%   beforeskip=-1sp,
%   afterskip=2\baselineskip]{chapter}

\pagestyle{plain}


% Fonts
% \usepackage[T1]{fontenc}
% \usepackage{unicode-math}
% \setmainfont{cm}
% \usepackage{tipa}

% \usepackage{ebgaramond}

\usepackage{fontspec}
% \fontspec[Path = ./fonts/] { }

\setmainfont
[Path = ./fonts/,
%
% Scale=1,
% BoldFont = texgyrepagella-bold.otf,
% ItalicFont = texgyrepagella-italic.otf,
% BoldItalicFont = texgyrepagella-bolditalic.otf]
% {texgyrepagella-regular.otf}
%
% BoldFont = BaskervilleMTStd-Bold.otf,
% ItalicFont = BaskervilleMTStd-Italic.otf,
% BoldItalicFont = BaskervilleMTStd-BoldIt.otf]
% {BaskervilleMTStd-Regular.otf}
%
BoldFont = ACaslonPro-Bold.otf,
ItalicFont = ACaslonPro-Italic.otf,
BoldItalicFont = ACaslonPro-BoldItalic.otf]
{ACaslonPro-Regular.otf}
%
% BoldFont = SabonLTStd-Bold.otf,
% ItalicFont = SabonLTStd-Italic.otf,
% BoldItalicFont = SabonLTStd-BoldItalic.otf]
% {SabonLTStd-Roman.otf}
%
% BoldFont = MinionPro-Bold.otf,
% ItalicFont = MinionPro-It.otf,
% BoldItalicFont = MinionPro-BoldIt.otf]
% {MinionPro-Regular.otf}
%
% BoldFont = BemboStd-Bold.otf,
% ItalicFont = BemboStd-Italic.otf,
% BoldItalicFont = BemboStd-BoldItalic.otf]
% {BemboStd.otf}

\author{\textsc{Oğuz Atay}}
\title{TUTUNAMAYANLAR}
\date{}

\begin{document}
\maketitle

\chapter*{Önsöz}

Şaksiper Kimdir, Eseri Nedir? Yıllar önce yayımlanmış bir broşürün adıydı bu. Ne
yazık ki artık adını hatırlayamadığım müellifi, ünlü İngiliz yazarını şöyle
15-20 sayfalık küçük ama yoğun bir broşürle anlatıyordu. Kitapçığın kapağında
``Şaksiper''in resmi bile vardı.

Oğuz Atay'ın hayatını ve eserlerini kapsayan bir önsöz yazmak çabası da işte bu
``adsız'' araştırmacınınki kadar acıklı ve tuhaf görünüyor bana. Zaten Oğuz
Atay'ın kendisi de, pek çok kurumun yanı sıra ``önsöz''leri tefe koyup yerlebir
etmişti. Biraz sonra ayrıntısıyla okuyacağınız gibi, Tutunamayanlar'da şöyle
diyor meselâ:

```Hayatı ve eserleri'. Hiç bıkmıyorum bunları tekrar tekrar okumaktan. Yazarın
her kitabını okurken `Hayatı ve Eserleri' yeniden karşıma çıkıyor. Bir daha, bir
daha okuyorum. Sanki önceden `Hayatı ve Eserleri'ni bilmiyormuş gibi yapıyorum:
yeni baştan heyecanlanmak için. Yalnız, yazarlar arasında bir birlik bulunmaması
beni yoruyor. Hiç olmazsa önsözleri yazanlar, yılda bir kere toplanmalı ve
aralarında ortak esaslar tespit etmeli. Bugünkü durum esef verici. Bakıyorsun
bir yazar, çok zor birleştiriyor kelimeleri. Bir türlü cümleleri kuramıyor. Öyle
diyor önsöz amca. Geçer karatahtanın başına diyor, yazar, bozar, uğraşır. Bütün
bunları da yarı karanlıkta yapar. İstediği cümleyi bulunca da koşar, bütün
ışıkları yakar. Ben de tam bu üstadın huylarını benimsemek üzereyken, bir önsöz
daha geçiyor elime. Bu önsöz de yazarın coşkun bir ırmak gibi yazdığını
anlatıyor. Kendisini tutamıyor bu adam: bıraksan günde yüz sayfa yazacak. (...)
Kime hizmet edeceğimi şaşırıyorum. Onlara uşaklık etmekte zorluk çekiyorum. Biri
İnsanlardan kaçıyor, öteki bir dakika yalnız kalamıyor. Sonunda hükümet el
koyacak bu işe. Hepsine haddini bildirecek. Bizi zehirlemeye ne hakları var.''

Herhalde bana katılıyorsunuz artık, Oğuz Atay'a önsöz yazmak, başa çıkılacak
gibi bir iş değil. Öte yandan, insanın imzasının Oğuz Atay'ın kitabında yer
almasının vereceği hoşluk duygusu ve bir de `ben bu işin altından kalkarım nasıl
olsa' düşüncesiyle (ya da düşüncesizliğiyle) ön sayfalar için önceden yer
ayırtmış olmanın verdiği kaçınılmazlık var.

Oğuz Atay ölümünden önceki yedi yıl süresince âdeta hummalı bir tempoda
çalışarak birbiri ardından çeşitli dallarda birçok eser verdiyse de, sağlığında
yeterince tanınmıştı denemez. Hatta, ölümünden altı yıl sonra günışığına
çıkarılan o eşsiz günlüğünde kendisinin daha ``yaşarken unutulduğu''nu söylüyor.
Bunun sebebi şu olabilir:

Aydın ``sınıfı'' içinde yer alıyordu Oğuz Atay ve o sınıfın derinlemesine
tahlilini yapıyordu. Biraz karamsar, biraz acı, çokça güldürücü... aydınsı bir
tahlil işte. Tüm romanlarının, öykülerinin ve oyununun ana konusunu bu meselenin
oluşturduğu söylenebilir. Öylesine bir irdelemedir ki bu, sonunda ortaya hiç de
küçümsenmeyecek boyutta bir ``aydınlar destanı'' çıkmıştır. Hatta bir ``aydınlar
marşı'' değilse bile, tutmuş, ``Şarkılar''ı yazmıştır Oğuz Atay. Durum bu noktada
çatallaşmaktadır işte. Bütün bunları okuyan ``aydın''lar, birdenbire billur bir
boy aynasında çırılçıplak buluverirler kendilerini. Korkunç bir durum canım,
utanç verici. Üstüne üstlük, sahnenin ortalık yerine öyle durup dururken pat
diye düşüveren bu mühendis bozması çok da iyi yazmıyor muydu size. Alın bakalım.

Kötü bir rüya gibi bir şeydi bu. Bir karabasan bile sayılabilirdi. Gözlerimizi
derhal yeniden yumup, rüyasız, deliksiz, hasetsiz, yeniden uyumaktan başka çare
yoktu. Allah hepimize rahatlık versin.

Verdi de. Tutunamayanlar'ın ilk cildinin basılmasının üzerinden on beş yıla
yakın zaman geçti. Yaşantının fazlasıyla yoğun ve `olaylı' geçtiği bizimki gibi
ülkelerde on beş yıl bir ömre bedeldir.

İşte bir ömrü tamamladık biz de ve işte yeni ve sağlıklı bir kuşak yetiştirdik
bu arada pırıl pırıl umutlarla. Sağlıklı diyorum, çünkü Oğuz Atay'ın eseriyle
ilgili bir de sağlıksız değerlendirme yapılıyordu gibime geliyor: Yani onun her
satırına sinmiş olan korkunç mizahı, sanıyorum bazı insanlarda derin bir
inançsızlığa, cynique bir tavra kaynaklık etti. Bunda biraz haksızlık var. Çünkü
Oğuz Atay ne denli karamsar ve ölümle içiçe olursa olsun, her zaman biraz umudu
barındırır içinde; ``canım insanlar'' der hep. Yani, Batıcıl aydın tipini
bulamazsınız onda, soğuk, alaycı ve hepten inançsız. Onun kitaplarını daha
sağlam bir değerlendirmeye tabi tutmak için de artık yeterli bir süre geçmiş
olduğu umulur.

İşte bütün bu sebeplerden ve başka sebeplerden dolayı, ölümünden yedi yıl sonra
Oğuz Atay'ın yaşama şansı çok artmış durumda.

Bu `önsöz'ü, Atay'ın Günlük'ünü günlük bir gazetede yayımladığımız sıralarda
yazdığım `önsöz'den bir parça ile bitirmek istiyorum:

O, ömür boyunca hep ``acele etmiş''tir; bu yüzden de hep ``geç kalmış''tır. Sürekli
bir panik vardır hayatında: Bir kitap okur, bir komedi seyreder, yorulur.
Birileriyle birlikte olur, derdini anlatamaz, telâşlanır ve incinir. Küçük
dertler, biryerlere ödenmesi gereken paralar, bazı şeylerin tamir masrafları hiç
eksik olmaz ve bu panik duygusuna katkıda bulunurlar. Ve hep acele edilir.

Bu acele içinde ölümden mi kaçılıyordur, yoksa kovalanıyor mudur ölüm, orası pek
belli değildir. Öyle bir kaçmakovalamaca oyunu işte.

Ve işte böyle çılgınca koşuştururken Oğuz, sırtından hiç çıkartmadığı mizah
zırhının tangırtısı da dünyayı tutar.

Nefes nefese koşarken bize hepimizin derdini anlatmak için üç roman, bir öykü
kitabı, bir oyun, bir de şu ``kırık'' günlüğü, yani beşbuçuk yapıtı bırakan bir
adamı unutmak birçok kişinin işine gelebilir belki, ama onu ``unutturmak'', işte o
biraz zor olabilir. Ne yapılsa nafile bence; perde yeniden açılıyor işte.

Oğuz Atay, gerçeğin bağrından filizlenen oyundan, oyunun uzandığı ölümden, ölüm
duyusundan doğan yaşantı damlasından, gözyaşında titreşen çılgın kahkahadan,
delilikte tüneyen akıldan, akıldan türeyen gönülden örülmüş o çok gülünçlü ve
çok acıklı dünyası ile Türk aydınını ve her şeyi yeniden kapsayacaktır yakında.

\begin{flushright}
  ÖMER MADRA

  Şubat 1984, İstanbul
\end{flushright}


% \flushleft
\part*{Birinci Bölüm}

\chapter{}

Olay, XX. yüzyılın ikinci yarısında, bir gece, Turgut'un evinde başlamıştı. O
zamanlar daha Olric yoktu, daha o zamanlar Turgut'un kafası bu kadar karışık
değildi. Bir gece yarısı evinde oturmuş düşünüyordu. Selim, arkasından bir de
herkesin bu durumlarda yaptığı gibi, mektuba benzer bir şey bırakarak, bu
dünyadan birkaç gün önce kendi isteğiyle ayrılıp gitmişti. Turgut, bu mektubu
çalışma masasının üstüne koymuş, karşısında oturup duruyordu. Selim'in titrek
bir yazıyla karaladığı satırlar gözlerinin önünde uçuşuyordu. Harflerin arasında
arkadaşının uzun parmaklarını seçer gibi oluyor, okuduğu kelimelerle birlikte
onun kalın ve boğuk sesini duyduğunu sanıyordu.

O zamanlar, henüz, Olric yoktu; hava raporları da günlük bültenlerden sonra
okunmuyordu. Henüz durum, bugünkü gibi açık ve seçik, bir bakıma da belirsiz
değildi.

``Bu mektup, neden geldi beni buldu?'' diye söyleniyordu hafifçe. Demek, hafifçe
söylenme alışkanlığı, o zamana kadar uzanıyordu. Demek, kendi kendine konuşma o
gece yarısı başlamıştı. Çevresindeki eşyaya duyduğu öfkenin ifade edilemeyen
sıkıntısıyla bunalıyordu. Selim, belki bu yaşantıyı, önde bir salonsalamanje,
arkada iki yatak odası, koridorun sağında mutfaksandık odasıbanyo, içerde uyuyan
karısı ve çocukları, parasıyla orantılı olarak yararlandığı küçük burjuva
nimetleri onu, nefes alamaz bir duruma getirmişti diye tanımlayabilirdi. Turgut,
anlamsız bakışlarla süzüyordu çevresini henüz. Duvarlar, resim yaptığı dönemden
kalma `eserler'le doluydu. Nermin çerçeveletmiş hepsini; benimle öğünüyor.
``Resimlerini çerçeveletmişsin, iyi olmuş,'' demişti Selim. ``Ben değil, karım,''
diye karşılık vermişti. Karısı odada yoktu. Bir resim aşağıda, bir resim
yukarıda; bir duvar resimle doldurulmuş, bir duvarın yarısı boş: simetriyi
bozmak için. Efendim? Efendim, derdi Selim olsaydı son heceye basarak. Ev sahibi
de kızmıştı duvarların bu renge boyandığını görünce ama belli etmemişti. Tavana
kadar aynı renk, böylece düzlemler daha kesin beliriyor, modern sanatın burjuva
yaşantısına katkısı. Efendim? Oysa, ne güzeldi eskiden: tavana bir karış kala,
bir parmak kalınlığında koyu renk, yatay bir çizgi çizilirdi; duvarın rengi
orada biterdi işte. Selimlerin Ankara'daki evinde öyleymiş. Tek parti devrinin
kalıntısı, fazla askeri bir düzen. O günlerde tavana kadar yükselen kitaplıklar
yoktu herhalde; yatay çizgi kaybolurdu kitapların arkasında böyle olsaydı.

İsteksiz bir kımıldanışla yerinden kalktı, kitaplığının karşısına geçti. Selim'e
özenerek alınan kitaplar; yüzlerce kitap, çoğu hiç okunmamış duruyordu öylece.
``Hiç evden çıkmadan beş yıl sürekli okusan, belki biter bu kitaplar,'' demişti
Selim. Ne demek? İçinde birden, hepsini okuyup bitirme ateşi yandı: kitapları
her görüşünde yanan eski ateş. Kaç sayfa eder hepsi? Bin sayfa, beş bin sayfa,
on bin sayfa. Bir sayfa kaç dakikada okunur, yemek ve uyku saatleri çıkarılırsa
geriye günde kaç saat kalır, cumartesi, pazar ve bayramlar için daha uzun süre
konursa... istersem yutarım hepsini. Okuldaki günleri aklına geldi: böyle,
hırsla eline aldığı kitapların beş on sayfasını okuduktan sonra içinin bir balon
gibi söndüğünü hatırladı. Bir kitabı bırakır ötekine saldırırdı. Bu ümitsizce
çırpınış, bütün kitapların yüzüstü bırakılmasıyla sona erer, büyük bir utanç ve
hayata dönüş buhranları gelirdi arkasından.

Kitaplığının önünden zorla ayırdı kendini: oyuna gelmeyelim yeniden. Aynı
zamanda yatak olabilen kanepeye oturdu ve bir düğmeye basınca içinden sahte
ağızlıklara sokulmuş sigaralar çıkan kutudan bir sigara alıp Alâettin'in lâmbası
biçimindeki çakmakla yaktı. Durum, ümit verici değildi: yerdeki halı,
mobilyalara hiç uymuyordu. Düğün hediyesi. Ne yapalım, istediğimiz gibi halı
alacak paramız yoktu. Sigarasını, yaprak biçimi gümüş tablada söndürdü. Karım
kızacak. Bu tablalar neden duruyor öyleyse? Bilinmez. Çalışma masasına yaklaştı.
Kaya'nın ayrı bir çalışma odası var. Orada ne çalışıyor? Bilinmez. Ben ne
çalışıyorum? Mektubu okuyorsun ya! Öyle ya. Selim'in yazdığı satırlara eğildi
yeniden.

Olay, böyle bir ortamda başlamıştı. Aslında, buna olay bile denemezdi. Turgut,
yani bir bakıma bir zamanlar onun en iyi arkadaşı, olayı gazeteden, yani
olayları veren bir `organ'dan öğrendiği için, olay diye adlandırılabilirdi bu
durum. Turgut yeni uyanmıştı: her sabah kapıcının kapının altından attığı
gazetenin hışırtısını bekliyordu. Sesi duyunca, karısını uyandırmamaya
çalışarak, uyuşuk hareketlerle terliklerini aramış, sonra, yavaşça `olay'a doğru
bilmeden yönelmişti. Yedinci sayfada, bir cinayet haberinin sonunu ararken
birden çarpmıştı `olay' gözüne. Sonra karısı, yatakta sarılarak onu teselli
etmişti. Bu gece de erken yattı beni rahatsız etmemek için; rahmetliyi dilediğim
gibi düşünebilmem için. Kendine düşeni yaptı fazlasıyla. Erken yatmasının başka
bir nedeni de yarınki direksiyon kursu. Ben de yatıp uyumalıyım; herkes yatıp
uyumuştur. Benden başka kimse, bu mektubun anlamını düşünmüyor. Kaya şimdi
çalışma odasında olsaydı ne yapardı? Üniversiteli kızların soyunmasını
seyrederdi. Hele bir tanesi varmış; her gece, her gece bacaklarını duvara
dayayıp... Karısından gizli, yani kaçamak. Ben de kaçamak yapıyorum şimdi:
karımdan gizli, Selim'i düşünüyorum. Hayır, gizli değil; biliyor kimi
düşündüğümü. Gene de bir gizlilik var: ne düşündüğümü, nasıl düşündüğümü
bilmiyor. Selim'i ve kızların bacaklarını... Selim de olsaydı seyrederdi, ben de
seyrederdim. Olmuyor; düşünce suçları, kaçamaklar artıyor. Ayağa kalktı,
salondan çıktı, koridorun duvarına tutunarak karanlığı geçti. Yatak odasının
kapısını itti; uyuyan karısını seyretti ışığı yakmadan. ``Hayır, hayır.'' İpek
yorgan hışırdadı, karısı uyanır gibi oldu. ``Uyusaydın artık,'' diye mırıldandı,
yorganın içinden. ``Biliyorsun...''

Biliyordu: kaçamak sona ermeliydi artık. Turgut, o sırada tehlikeyi göremiyordu:
gene de bitmesi gerektiğini seziyordu bu olaya olan ilgisinin. Kaya'nın, karşı
binadaki yarı aralık kırmızı perdelerin arkasını merak etmesinden öte, daha
büyük bir tehlikeydi bu. Çıplak bir bacağın görüntüsüyle yatışan ilgiden daha
keskin bir şey: bir düşünce, geriye doğru giden bir merak. Selim olsa, sabaha
kadar uyumaz, düşünür dururdu. Ben olsam yatardım. Üniversitede okurken de ben,
gece yarısı olunca yatardım; o, çalışmasını sabaha kadar sürdürürdü. ``Saçların
dökülüyor, uykusuz çalışmaya dayanamıyorsun; oğlum Turgut, ihtiyarlıyorsun.''
``Uykusuz kalabilmen sinir kuvvetinden. Benimki adale kuvveti.'' Kollarıyla
Selim'i soluksuz bırakıncaya kadar sıkardı: ``Sen birden çökeceksin Selim. Çünkü
neden? Çünkü için boş senin. Birden, kollarımın arasında için boşalacak: birden,
üçüncü boyutunu kaybedip bir düzlem olacaksın ve ben de seni duvarda bir çiviye
asacağım.'' Havaya kaldırdığı Selim'i duvara sürüklerdi. Siyah saçlarından
yakalayarak başını duvara dayar: ``Dökülmeyen saçlarından asacağım seni,'' diye
bağırırdı. ``Erkeğin kılları göğsündedir, oğlum Selim.'' Hemen gömleğini çıkarır
ve boynuna kadar bütün gövdesini kaplayan kıllarını gösterirdi Selim'e.
``İğrençsin Turgut. Sen onları, üniversite kantinindeki kızlara göster. Kapat şu
ormanı.'' Bir erkeğin yanında soyunmasından sıkılırdı Selim. ``Beni, aşağılara
çekiyorsun Turgut. Senden kurtulmalıyım.'' Turgut, pantalonunu da çıkarır,
kollarını açarak bağırırdı: ``Ben, senin bilinçaltı karanlıklarına ittiğin ve
gerçekleşmesinden korktuğun kirli arzuların, ben senin bilinçaltı ormanlarının
Tarzan'ı! yemeye geldim seni. Benden kurtulamazsın. Ben, senin vicdan azabınım!''
``Bağırma, anladık. Benim vicdan azabım bu kadar kıllı olamaz. Ruhbilimci Tarzan,
lütfen giyin.''

Karısına karşılık vermeden yavaşça yatak odasından çıktı, kapıyı kapadı.
Koridorda yürürken kollarını havaya kaldırdı: ``Esir, Selim, esir,'' diye
mırıldandı. Selim'in, zevkle bağıran sesini duyar gibi oldu: ``Yenildin demek,
koca ayı. Evet, yenildin. Bu yenilginin tarihini hep birlikte bir kez daha
yaşıyoruz. Kurtuluş Savaşı'nın ateş ve dehşet dolu günlerinden biriydi.
Mühendishane'yi Berrii Hümayun'un üçüncü sınıfında talebeyken gönüllü olarak
askere yazılan genç mülazim Selim Efendi, Afyon dolaylarında, Kartaltepe
mevkiinde, tek başına mevzilenmişti. Düşman kurnaz bir kalabalıktı.
Mülazımıevvel Selim, boynunda bir kayışla asılı duran dürbünü eldivenlerini
çıkarmadan eline aldı; gözüne götürdüğü bu optik aletin okülerlerini iki
parmağının iki zarif hareketiyle çevirerek, görüş alanı içine aldığı düşmanın
görüntüsünü netleştirdi. Artık bütün hazırlıkları tamamdı; düşman hatlarını
gözetliyordu. Üsküdar'da, Soğanağası'nda, minimini bir çocukken ahşap
konaklarının tavan arasında hayal etmiş olduğu an, nihayet gelip çatmıştı.
`Sadece üç bin kişi', diye söylendi. Sonra, Tarzan gibi `Uuu..' diye üç kere
bağırdı, yumruklarıyla göğsünü dövdü. Düşman neye uğradığını şaşırmıştı.
Silahlarını yere atarak kaçıyorlardı. Askerin başındaki Yunan zabiti, Türkün, bu
gücünü göstermesi karşısında, yerinden bile kımıldayamamıştı; kollarını havaya
kaldırdı. Avuçlarının içinde eldivenin kapamadığı iki delikten teni
görünüyordu.'' Turgut: ``Yesir, yesir...'' diye bağırdı. Bir yandan da işaret
parmağıyla, muhayyel eldivenin boş bıraktığı avuç içi derisini gösteriyordu.

L biçimi salona döndü, maroken taklidi plastikle kaplı rahat koltuğuna oturdu;
bir düğmeye basarak koltuğu geriye itti. Yakalandın Turgut, kendini eleverdin.
Neden, Selim? Nasıl olur, tam şirketin muhasebecisinden onbinpeşin yirmibeşbine
bir araba almak üzereyken, tam direksiyon kursuna başlayacakken, tam bir kat
parası biriktirmenin gerekliliğini düşünürken... beni kandıramazsın Selim, işime
burnunu sokamazsın. Ben, soğukkanlılığımı korumasını bilirim. Sen söylemez
miydin `utanmadan, duygusuzluğumla öğündüğümü'. On yıl önce olsaydı, belki biraz
daha düşünürdüm; belirsiz tehlikelerden korkmazdım. On yıl önce olsaydı,
Oblomov'u okuduktan sonra beden hareketlerine başlamam gibi, gene bu sarsıcı
olayla kımıldardım yerimden belki. Kımıldardım da ne yapardım? Hiç. Biraz
huzursuzluk duyardım herhalde. Eski bir yara yerinin sızlaması gibi bir şey.
Oblomov'u ve beden hareketlerini unuttum. Kendimi çabuk toparladım. Bilinmeyen
yüz binlerce kız içinde, üniversite kantininden birini seçtin kendine ve ona
okuduğu kitapları sordun ve karşında oturup susmasını seyrettin. Evet; öyle oldu
Selim; ne kötülük görüyorsun bu davranışımda? Bir şey dediğim yok, Turgut.
Evlenirken de bir şey söyledim mi? Bize çok uğramadın evlendikten sonra. Size
mi? Siz kimsiniz? Ben, Nermin, çocuklar... Ben sizi bilmiyorum, seni tanıyorum.
Evinize alışamadım herhalde. Eşyalarınıza alışamadım, yadırgadım onları.
Salonsalamanjeyi, deniz gibi büyük ve kauçuk köpüklü yatağı olan karyolayı, aynı
takımın yaldızlı gardrobunu ve gene aynı takımın şifonyerini ve gene aynı
takımın tuvaletini sevemedim. Evinizde Türkçe bir şey kalmamıştı. Bana anlayış
gösterecek yerde büfeyi gösterdin. Kelime oyunu yapıyorsun Selim. Benim bütün
işim oyundu, bunu biliyorsun Turgut. Hayatım, ciddiye alınmasını istediğim bir
oyundu. Sen evlendin ve oyunu bozdun. Bütün hayatımca nasıl oynayabilirdim? Sen
de dayanabildin mi? Sen de ürkütücü bir gerçekle bozdun bu oyunu. Herkesin
belirli bir işle uğraştığı bu kocaman dünyada yalnız başına oradan oraya
sürüklendin canım kardeşim benim. Necati'nin işi oyun yazmaktı. Küçük burjuva
alışkanlıklarını yeren son oyununu hatırlıyor musun? Oyunun yarısında çıkmıştım.
Sen bütün oyunların yarısında çıktın aslında. Necati'nin oyunu dört yüz elli
kere oynandı ve Necati de bir kat aldı kazandığı parayla. Senin işin neydi
onların arasında? Ne yapıyordun? Hiçbir işim yoktu. Bu nedenle sevmezlerdi seni
işte. Bu nedenle aldırmadılar sana. Senin ne işin vardı orada? Herkesin işine
karıştın, işin olmadığı halde. Ölmek bile, kendilerine böyle bir görev
verilenlerin işidir. Kendine oyunlar buldun: başkalarının katılıp katılmadığına
aldırmadığın oyunlar. Herkesi yargıladın bu oyunlarda. Bu arada beni de
yargıladın, bana da haksızlık ettin. Ben de bir oyun yazsam, sonunda haklı
çıkmak için kendini öldürdüğünü söylesem... Bu oyunu sevmedim Turgut. Ben,
oyunlarda bana saldırılmasını sevmem. Ben oyun istemiyorum artık; ne oyun ne de
gerçek, senin ölmen gibi bir gerçek, beni sarsmamalı Selim. Ayağa kalktı. İnsan
gerçeklere karşı durur: yaşar ve olduğu gibi olmayı sürdürür Selim. Ayrıca, bu
mektubu bana yollamadın, bana böyle bir görev verilmedi. Benim işim değil bu.
Benim işim değil. Mektubunu on kere okudum, bir sonuca varamadım. Başka türlü
bir yaşantın olabilirdi Selim. Seni istemeyenlerin dışında bir düzen
kurabilirdin.

``Bu sözlerimle belki birşeyler kaybediyorum Selim,'' diye yüksek sesle söylendi.
Saat üçe geliyordu; Turgut'un kafası karışıyordu. Olayın iyi başlamadığını
seziyordu. Neye göre iyi? Bilemiyordu. ``Benim işim değil,'' diye mırıldanarak
yatak odasına doğru yürüdü.

\section*{}




\end{document}

